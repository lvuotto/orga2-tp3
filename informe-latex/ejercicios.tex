\subsection*{Ejercicio 1}
\vspace*{0.5cm} 

\noindent
a) Completar la Tabla de Descriptores Globales (\verb|GDT|) con 4 segmentos, dos para código de
nivel 0 y 3; y otros dos para datos de nivel 0 y 3. Estos segmentos deben direccionar los
primeros 733MB de memoria. En la \textit{gdt}, por restricción del trabajo práctico, las primeras 
8 posiciones se consideran utilizadas y no deben utilizarse. El primer índice que deben usar
para declarar los segmentos, es el 9 (contando desde cero).

\vspace*{0.3cm}

\textbf{completar.}

\vspace*{0.5cm} \noindent



\noindent
b) Completar el código necesario para pasar a modo protegido y setear la pila del \textit{kernel}
en la dirección \verb|0x27000|.

\vspace*{0.3cm}

\textbf{completar.}

\vspace*{0.5cm} \noindent



\noindent
c) Declarar un segmento adicional que describa el área de la pantalla en memoria que pueda ser 
utilizado sólo por el \textit{kernel}.

\vspace*{0.3cm}

\textbf{completar.}

\vspace*{0.5cm} \noindent



\noindent
d) Escribir una rutina que se encargue de limpiar la pantalla y pintar en el área de \textit{el_mapa} un 
fondo de color (sugerido verde). Para este ejercicio se debe escribir en la pantalla usando el segmento 
declarado en el punto anterior (para los próximos ejercicios se accederá a la memoria de video por medio del 
segmento de datos de 733MB).

\vspace*{0.3cm}

\textbf{completar.}

\vspace*{0.5cm} \noindent



\newpage





\subsection*{Ejercicio 2}
\vspace*{0.5cm}

\noindent
a) Completar las entradas necesarias en la \verb|IDT| para asociar diferentes rutinas a todas las
excepciones del procesador. Cada rutina de excepción debe indicar en pantalla qué problema se produjo
e interrumpir la ejecución. Posteriormente se modificarán estas rutinas para que se continúe la ejecución,
resolviendo el problema y desalojando a la tarea que lo produjo.

\vspace*{0.3cm}

\textbf{completar.}

\vspace*{0.5cm} \noindent



\noindent
b) Hacer lo necesario para que el procesador utilice la \verb|IDT| creada anteriormente. Generar una
excepción para probarla.

\vspace*{0.3cm}

\textbf{completar.}

\vspace*{0.5cm} \noindent



\newpage





\subsection*{Ejercicio 3}
\vspace*{0.5cm}

\noindent
a) Escribir una rutina que se encargue de limpiar el \textit{buffer} de video y pintarlo como indica la figura 8.
Tener en cuenta que deben ser escritos de forma genérica para, posteriormente, ser completados con información del
sistema. Además, considerar estas imágenes como sugerencias, ya que pueden ser modificadas a gusto según cada grupo,
mostrando siempre la misma información.

\vspace*{0.3cm}

\textbf{completar.}

\vspace*{0.5cm} \noindent



\noindent
b) Escribir las rutinas encargadas de inicializar el directorio y tablas de páginas para el \textit{kernel}\newline
(\verb|mmu_inicializar_dir_kernel|). Se debe generar un directorio de páginas que mapee, usando \textit{identity mapping}, 
las direcciones \verb|0x00000000| a \verb|0x00DC3FFF|, como ilustra la figura 5. Además, esta función debe inicializar
el directorio de páginas en la dirección \verb|0x27000| y las tablas de páginas según muestra la figura 1.

\vspace*{0.3cm}

\textbf{completar.}

\vspace*{0.5cm} \noindent



\noindent
c) Completar el código necesario para activar paginación.

\vspace*{0.3cm}

\textbf{completar.}

\vspace*{0.5cm} \noindent



\noindent
d) Escribir una rutina que imprima el nombre del grupo en pantalla. Debe estar ubicado en la primer línea
de la pantalla, alineado a la derecha.

\vspace*{0.3cm}

\textbf{completar.}

\vspace*{0.5cm} \noindent



\newpage





\subsection*{Ejercicio 4}

\noindent
a) Escribir una rutina (\verb|inicializar_mmu|), que se encargue de inicializar las estructuras necesarias para administrar la memoria en el área libre.

\vspace*{0.3cm}

\textbf{completar.}

\vspace*{0.5cm} \noindent



\noindent
b) Escribir una rutina (\verb|mmu_inicializar_dir_tarea|), encargada de inicializar un directorio de páginas y tablas de páginas para una tarea, respetando la figura 5 (ver enunciado). La rutina debe copiar el código de la tarea a su área asignada, es decir, sus dos páginas de código dentro de \textit{el_mapa} y mapear dichas páginas a partir de la dirección virtual (\verb|0x08000000|) (128MB).

\vspace*{0.3cm}

\textbf{completar.}

\vspace*{0.5cm} \noindent



\noindent
c) Escribir dos rutinas encargadas de mapear y desmapear páginas de memoria.

\begin{enumerate}
  \item \verb|mmu_mapear_pagina(unsigned int virtual, unsigned int cr3, unsigned int fisica)| \newline
  Permite mapear la página física correspondiente a \verb|fisica| en la dirección virtual \verb|virtual| utilizando \verb|cr3|.
  
  \item \verb|mmu_unmapear_pagina(unsigned int virtual, unsigned int cr3)| \newline
  Borra el mapeo creado en la dirección virtual \verb|virtual| utilizando \verb|cr3|.
\end{enumerate}

\vspace*{0.3cm}

\textbf{completar.}

\vspace*{0.5cm} \noindent



\noindent
d) Construir un mapa de memoria para tareas e intercambiarlo con el del \textit{kernel}, luego cambiar el color del fondo del primer caracter de la pantalla y volver a la normalidad.

\vspace*{0.3cm}

\textbf{completar.}

\vspace*{0.5cm} \noindent



\newpage





\subsection*{Ejercicio 5}

\vspace*{0.3cm} \noindent

completar.

\newpage





\subsection*{Ejercicio 6}

\vspace*{0.3cm} \noindent

completar.

\newpage





\subsection*{Ejercicio 7}

\vspace*{0.3cm} \noindent

completar.

\newpage





\subsection*{Ejercicio 8 (optativo)}

\vspace*{0.3cm} \noindent

completar.

\newpage
