\subsection*{Ejercicio 1}
\vspace*{0.5cm} 

\noindent
a) Completar la Tabla de Descriptores Globales (\verb|GDT|) con 4 segmentos, dos para código de
nivel 0 y 3; y otros dos para datos de nivel 0 y 3. Estos segmentos deben direccionar los
primeros 733MB de memoria. En la \textit{gdt}, por restricción del trabajo práctico, las primeras 
8 posiciones se consideran utilizadas y no deben utilizarse. El primer índice que deben usar
para declarar los segmentos, es el 9 (contando desde cero).

\vspace*{0.3cm}

En \verb|gdt.c|, creamos un arreglo \verb|gdt|, el cual contiene 13 elementos (en este instante del tp, ya 
que más adelante agregaremos otras entradas para la TSS y la cantidad total ascender\'a a 17).  
Las entradas de este arreglo son las siguientes: 

\begin{itemize}

  \item \verb|GDT_IDX_NULL_DESC|: Esta es la entrada nula, 
necesaria para el correcto funcionamiento del mecanismo de la \verb|GDT|.

  \item Las entradas 1 a 8 est\'an reservadas por restricciones propias del enunciado del tp. 
  El tama\~no de las entradas GDT_IDX_CD_0, GDT_IDX_CD_1, GDT_IDX_DD_0, GDT_IDX_DD_1 y es 
  m\'ultiplo de 4KB, valiendo 733MB.
  
  Para los segmentos mencionado anteriormente, usamos \verb|0x0000| como base y \verb|0xdcff|, es 
  decir, direccionamos 733MB utilizando segmentación \textit{flat}, de manera que los mismos se solapan. 
  Además, seteamos el bit de \textit{granularity} en 1, ya que todos estos segmentos van a ocupar más de 1MB.

  \item \verb|GDT_IDX_CD_0|: Esta entrada contiene la descripci\'on del segmento de c\'odigo de nivel 0,
  que ser\'a utilizado para ejecutar el c\'odigo del \textit{kernel}. 

  \item \verb|GDT_IDX_CD_1|: Esta entrada contiene la descripci\'on del segmento de c\'odigo de nivel 3. 

  \item \verb|GDT_IDX_DD_0|: Esta entrada contiene la descripci\'on del segmento de datos de nivel 0.

  \item \verb|GDT_IDX_DD_1|: Esta entrada contiene la descripci\'on del segmento de datos de nivel 3.

\end{itemize}


\vspace*{0.5cm} \noindent



\noindent
b) Completar el código necesario para pasar a modo protegido y setear la pila del \textit{kernel}
en la dirección \verb|0x27000|.

\vspace*{0.3cm}

Para entrar en modo protegido, deshabilitamos las interrupciones \textit{(cli)}, habilitamos la 
\textit{(A20 Gate)}, creamos y cargamos la \textit{GDT} (previamente cargamos en el registro GDTR la 
dirección base y el límite de la GDT). Luego, activamos el bit correspondiente del \verb|cr0|, y 
ejecutamos la instrucci\'on \verb|jmp 0b1001000:protected_mode|. Para setear la pila del \verb|kernel|, ejecutamos 
la instrucci\'on \verb|mov esp, 0x27000|.

\vspace*{0.5cm} \noindent



\noindent
c) Declarar un segmento adicional que describa el área de la pantalla en memoria que pueda ser 
utilizado sólo por el \textit{kernel}.

\vspace*{0.3cm}

Agregamos la entrada \verb|GDT_IDX_SD|, que contiene la descripci\'on del segmento de video, del nivel 0, 
utilizado para la pantalla del juego. Este descriptor de segmento posee \textit{granularity} 0, ya que 
sólamente ocupa 32KB en memoria. Para que pueda ser utilizado \textit{sólo por el kernel}, seteamos el 
\textit{descriptor privilege level (dpl)} en 0.



\vspace*{0.5cm} \noindent



\noindent
d) Escribir una rutina que se encargue de limpiar la pantalla y pintar en el área de \textit{el_mapa} un 
fondo de color (sugerido verde). Para este ejercicio se debe escribir en la pantalla usando el segmento 
declarado en el punto anterior.

\vspace*{0.3cm}

\begin{itemize}
  
  \item \verb|limpiar_pantalla|: Esta rutina se basa en que el selector \verb|fs| apunta al comienzo del segmento de video \verb|0xb8000|. 
  Recorremos el \'area de 80x50 por fila y escribimos de a 2 caracteres vac\'ios (para optimizar la 
  implementaci\'on).

  \item \verb|pintar_pantalla|: Utilizamos un algoritmo muy similar al de \item \verb|pintar_pantalla|, s\'olo 
  que esta vez, en vez de escribir caracteres, pintamos el fondo (background) de verde.
 
\end{itemize}



\vspace*{0.5cm} \noindent



\newpage





\subsection*{Ejercicio 2}
\vspace*{0.5cm}

\noindent
a) Completar las entradas necesarias en la \verb|IDT| para asociar diferentes rutinas a todas las
excepciones del procesador. Cada rutina de excepción debe indicar en pantalla qué problema se produjo
e interrumpir la ejecución.

\vspace*{0.3cm}

Completamos las entradas de la \verb|IDT| utilizando un \textit{macro} provisto por la c\'atedra, con 
algunas modificaciones. El \verb|dpl| fue seteado en 0.
Las entradas est\'an definidas como \verb|Interrupt Gates|, y el selector de segmento es el de 
c\'odigo, utilizado por el kernel (entrada 9 de la \verb|GDT|, \verb|GDT_IDX_CD_0|). Una \verb|Interrupt Gate| se 
utiliza para especificar una rutina de interrupci\'on de servicio, deshabilita el llamado a futuras rutinas de 
atenci\'on de interrupciones, haci\'endola especialmente \'util, por ejemplo, para atender interrupciones de hardware.
Las rutinas de atención de las interrupciones se encuentran en \verb|isr.asm|.

Cada rutina muestra un mensaje alusivo al tipo de excepci\'on generada. Adem\'as, de producirse dicha 
excepci\'on, muestra por pantalla el estado de los registros.

\vspace*{0.5cm} \noindent



\noindent
b) Hacer lo necesario para que el procesador utilice la \verb|IDT| creada anteriormente.

\vspace*{0.3cm}

Para que el procesador utilice la \verb|IDT|, ejecutamos primero la instrucci\'on \verb|call idt_inicializar|, 
para inicializar la tabla, y luego \verb|lidt [IDT_DESC]|, para cargarla en el registro correspondiente.

\vspace*{0.5cm} \noindent



\newpage





\subsection*{Ejercicio 3}
\vspace*{0.5cm}

\noindent
a) Escribir una rutina que se encargue de limpiar el \textit{buffer} de video y pintarlo como indica la figura 8.

\vspace*{0.3cm}

Esta rutina simplemente escribe caracteres nulos en el segmento de video.

\vspace*{0.5cm} \noindent



\noindent
b) Escribir las rutinas encargadas de inicializar el directorio \textit{page directory} y tablas de páginas \textit{page tables} 
para el \textit{kernel}\newline (\verb|mmu_inicializar_dir_kernel|). Se debe generar un directorio de páginas que mapee, 
usando \textit{identity mapping}, las direcciones \verb|0x00000000| a \verb|0x00DC3FFF|, como ilustra la figura 5. 
Además, esta función debe inicializar el directorio de páginas en la dirección \verb|0x27000| y las tablas de páginas según muestra la figura 1.

\vspace*{0.3cm}

Para inicializar el directorio de p\'aginas, utilizamos la funci\'on \verb|mmu_inicializar_dir_kernel|, la cual, 
primero crea las 1024 entradas del directorio, con permisos de lectura/escritura y nivel de privilegio 0. Las 4 
primeras entradas se marcan como presentes. Adem\'as, en estas 4, se cambian las bases, apuntando a \verb|0x28, 0x29, 0x2a, 0x2b|, 
respectivamente. Luego, inicializamos las tablas correspondientes a cada una de las entradas del directorio, marcando 
las primeras 3524 como presentes y dejando las 572 restantes como ausentes (no llegamos a utilizar 4 page tables completamente). 
El directorio es cargado en la direcci\'on \verb|0x27000|, mediante \textit{identity mapping}.

\vspace*{0.5cm} \noindent



\noindent
c) Completar el código necesario para activar paginación.

\vspace*{0.3cm}

Primero cargamos el directorio de p\'aginas, utilizando las instrucciones 

\begin{verbatim}
mov eax, 0x27000
mov cr3, eax 
\end{verbatim}

de manera que quede ubicado en la direcci\'on \verb|0x27000|. Luego, activamos el bit de paginaci\'on en \verb|cr0|

\begin{verbatim}
mov eax, cr0
or eax, 0x80000000
mov cr0, eax 
\end{verbatim}


\vspace*{0.5cm} \noindent



\noindent
d) Escribir una rutina que imprima el nombre del grupo en pantalla. Debe estar ubicado en la primer línea
de la pantalla, alineado a la derecha.

\vspace*{0.3cm}

Utilizamos una rutina provista por la c\'atedra, para imprimir en modo protegido

\verb|imprimir_texto_mp|

que recibe como par\'ametros: un puntero al string a imprimir, la longitud del string, el color del texto y fondo, 
y por \'ultimo las coordenadas fila y columna, respectivamente.

\vspace*{0.5cm} \noindent



\newpage





\subsection*{Ejercicio 4}
\vspace*{0.5cm}

\noindent
a) Escribir una rutina (\verb|inicializar_mmu|), que se encargue de inicializar las estructuras necesarias para administrar 
la memoria en el área libre.

\vspace*{0.3cm}

\verb|mmu_inicializar| llama a \verb|mmu_inicializar_dir_kernel| y luego, por motivos organizativos, implementamos 
un ciclo for, que utilizamos posteriormente para la implementaci\'on de los \textit{syscalls}.

\vspace*{0.5cm} \noindent



\noindent
b) Escribir una rutina (\verb|mmu_inicializar_dir_tarea|), encargada de inicializar un directorio de páginas y tablas de páginas para 
una tarea, respetando la figura 5 (ver enunciado). La rutina debe copiar el código de la tarea a su área asignada, es decir, 
sus dos páginas de código dentro de \textit{el_mapa} y mapear dichas páginas a partir de la dirección virtual (\verb|0x08000000|) (128MB).

\vspace*{0.3cm}

\verb|mmu_inicializar_dir_tarea| opera de manera similar a \verb|mmu_inicializar_dir_kernel|, pero, a diferencia 
de esta \'ultima, los directorios no comienzan en posiciones arbitrarias, sino que \verb|mmu_inicializar_dir_tarea| 
obtiene posiciones de memoria libre del \textit{\'area libre}. Luego, se copia el c\'odigo de las tareas a alguna 
posici\'on aleatoria del mapa y mapeamos la direcci\'on \verb|0x8000000| a dicha direcci\'on del mapa. Esta funci\'on adem\'as 
devuelve el \verb|cr3| para poder utilizar la estructura de directorios creada.

\vspace*{0.5cm} \noindent



\noindent
c) Escribir dos rutinas encargadas de mapear y desmapear páginas de memoria.

\begin{enumerate}
  \item \verb|mmu_mapear_pagina(unsigned int virtual, unsigned int cr3, unsigned int fisica)| \newline
  Permite mapear la página física correspondiente a \verb|fisica| en la dirección virtual \verb|virtual| utilizando \verb|cr3|.
  
  \item \verb|mmu_unmapear_pagina(unsigned int virtual, unsigned int cr3)| \newline
  Borra el mapeo creado en la dirección virtual \verb|virtual| utilizando \verb|cr3|.
\end{enumerate}

\vspace*{0.3cm}

\begin{itemize}
  
  \item \verb|mmu_mapear_pagina(unsigned int virtual, unsigned int cr3, unsigned int fisica, unsigned int attr)|:
  Mediante el \verb|cr3| recibido por par\'ametro, accedemos al \'indice de la tabla del directorio correspondiente
  en base a la direcci\'on virtual. Si dicha entrada no se encuentra presente, creamos una nueva tabla. Cualquiera
  sea el caso, al tener la tabla, nuevamente accedemos al \'indice correcto mediante la direcci\'on virtual. Esta
  p\'agina se marca como presente, se le asignan los atributos de lectura/escritura y usuario/supervisor recibidos
  como par\'ametros y le asignamos como base la direcci\'on f\'isica que queremos mapear. Finalmente se ejecuta
  \verb|tlbflush| para invalidar la cach\'e de la \verb|TLB|.   

  \item \verb|mmu_unmapear_pagina(unsigned int virtual, unsigned int cr3)|: Funciona en manera an\'aloga a
  \verb|mmu_mapear_pagina|. Una vez determinado el \'indice de la tabla de directorios en base al \verb|cr3| 
  recibido como par\'ametro y la direcci\'on virtual, si la entrada \textbf{no} est\'a presente, no se hace nada. 
  Sino, se procede acceder al \'indice correcto de la tabla, de nuevo en base a la direcci\'on virtual, y se marca dicha 
  p\'agina como no presente, independientemente de su estado. De nuevo, se ejecuta \verb|tlbflush| con el mismo fin 
  que en \verb|mmu_mapear_pagina|.
  

\end{itemize}


\vspace*{0.5cm} \noindent




\newpage





\subsection*{Ejercicio 5}

\vspace*{0.5cm}

\noindent
a) Completar las entradas necesarias en la \verb|IDT| para asociar una rutina a la interupci\'on 
del reloj, otra a la interrupci\'on del teclado y por \'ultimo, una a la interrupci\'on de software \verb|0x52|.

\vspace*{0.3cm}

Al igual que en el \textbf{ejercicio 2.a}, agregamos 3 nuevas entradas en la \verb|IDT|, asociando las interrupciones 
de reloj (32) y teclado (33), asign\'andoles los mismos valores. La interrupci\'on de software \verb|(0x52)|, en cambio, 
se carga igual exceptuando el \verb|dpl|, que esta vez es seteado en nivel 3, de usuario, ya que esta interrupción
corresponde a los \textit{syscalls}, que son llamados por las tareas.

\vspace*{0.5cm} \noindent



\noindent
b) Escribir la rutina asociada a la interrucpi\'on del reloj, para que por cada \textit{tick} llame a
la funci\'on \verb|screen_proximo_reloj|. La misma se encarga de mostrar, cada vez que se llame la
animaci\'on de un cursor rotando en la esquina inferior derecha de la pantalla.

\vspace*{0.3cm}

Pudimos manejar interrupciones que no son excepciones y en cada ciclo de reloj
llamamos una función que actualiza el clock.

\vspace*{0.5cm} \noindent



\noindent
c) Escribir la rutina asociada a la interrucpi\'on de teclado de forma que si se presiona cualquier
n\'umero, se presente el mismo en la esquina superior derecha de la pantalla. El n\'umero debe ser
escrito en color blanco con fondo de color aleatorio por cada tecla que sea presionada.

\vspace*{0.3cm}

Al igual que en el ítem anterior, escribimos la rutina correspondiente, pero alteramos ligeramente
la funcionalidad, de manera tal que, además de mostrar el valor correspondiente en la esquina superior 
derecha de la pantalla, con el fondo aleatorio, también mostramos el contenido de los registros de la 
tarea correspondiente.




\newpage





\subsection*{Ejercicio 6}

\vspace*{0.5cm}

\noindent
a) Definir 3 entradas en la \verb|GDT| para ser usadas como descriptores de \verb|TSS|. Una ser\'a reservada 
para la \verb|tarea_inicial| y otras dos para realizar el intercambio entre tareas, denominadas \verb|TSS1| 
y \verb|TSS2| respectivamente.

\vspace*{0.3cm}

\textbf{completar.}

\vspace*{0.5cm} \noindent



\noindent
b) .

\vspace*{0.3cm}

\textbf{completar.}

\vspace*{0.5cm} \noindent



\noindent
c) .

\vspace*{0.3cm}

\textbf{completar.}

\vspace*{0.5cm} \noindent



\noindent
d) .

\vspace*{0.3cm}

\textbf{completar.}

\vspace*{0.5cm} \noindent





\subsection*{Ejercicio 7}

\vspace*{0.3cm} \noindent

completar.

\newpage
